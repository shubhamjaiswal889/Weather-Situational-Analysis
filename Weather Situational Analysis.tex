\documentclass[11pt]{article}

    \usepackage[breakable]{tcolorbox}
    \usepackage{parskip} % Stop auto-indenting (to mimic markdown behaviour)
    
    \usepackage{iftex}
    \ifPDFTeX
    	\usepackage[T1]{fontenc}
    	\usepackage{mathpazo}
    \else
    	\usepackage{fontspec}
    \fi

    % Basic figure setup, for now with no caption control since it's done
    % automatically by Pandoc (which extracts ![](path) syntax from Markdown).
    \usepackage{graphicx}
    % Maintain compatibility with old templates. Remove in nbconvert 6.0
    \let\Oldincludegraphics\includegraphics
    % Ensure that by default, figures have no caption (until we provide a
    % proper Figure object with a Caption API and a way to capture that
    % in the conversion process - todo).
    \usepackage{caption}
    \DeclareCaptionFormat{nocaption}{}
    \captionsetup{format=nocaption,aboveskip=0pt,belowskip=0pt}

    \usepackage[Export]{adjustbox} % Used to constrain images to a maximum size
    \adjustboxset{max size={0.9\linewidth}{0.9\paperheight}}
    \usepackage{float}
    \floatplacement{figure}{H} % forces figures to be placed at the correct location
    \usepackage{xcolor} % Allow colors to be defined
    \usepackage{enumerate} % Needed for markdown enumerations to work
    \usepackage{geometry} % Used to adjust the document margins
    \usepackage{amsmath} % Equations
    \usepackage{amssymb} % Equations
    \usepackage{textcomp} % defines textquotesingle
    % Hack from http://tex.stackexchange.com/a/47451/13684:
    \AtBeginDocument{%
        \def\PYZsq{\textquotesingle}% Upright quotes in Pygmentized code
    }
    \usepackage{upquote} % Upright quotes for verbatim code
    \usepackage{eurosym} % defines \euro
    \usepackage[mathletters]{ucs} % Extended unicode (utf-8) support
    \usepackage{fancyvrb} % verbatim replacement that allows latex
    \usepackage{grffile} % extends the file name processing of package graphics 
                         % to support a larger range
    \makeatletter % fix for grffile with XeLaTeX
    \def\Gread@@xetex#1{%
      \IfFileExists{"\Gin@base".bb}%
      {\Gread@eps{\Gin@base.bb}}%
      {\Gread@@xetex@aux#1}%
    }
    \makeatother

    % The hyperref package gives us a pdf with properly built
    % internal navigation ('pdf bookmarks' for the table of contents,
    % internal cross-reference links, web links for URLs, etc.)
    \usepackage{hyperref}
    % The default LaTeX title has an obnoxious amount of whitespace. By default,
    % titling removes some of it. It also provides customization options.
    \usepackage{titling}
    \usepackage{longtable} % longtable support required by pandoc >1.10
    \usepackage{booktabs}  % table support for pandoc > 1.12.2
    \usepackage[inline]{enumitem} % IRkernel/repr support (it uses the enumerate* environment)
    \usepackage[normalem]{ulem} % ulem is needed to support strikethroughs (\sout)
                                % normalem makes italics be italics, not underlines
    \usepackage{mathrsfs}
    

    
    % Colors for the hyperref package
    \definecolor{urlcolor}{rgb}{0,.145,.698}
    \definecolor{linkcolor}{rgb}{.71,0.21,0.01}
    \definecolor{citecolor}{rgb}{.12,.54,.11}

    % ANSI colors
    \definecolor{ansi-black}{HTML}{3E424D}
    \definecolor{ansi-black-intense}{HTML}{282C36}
    \definecolor{ansi-red}{HTML}{E75C58}
    \definecolor{ansi-red-intense}{HTML}{B22B31}
    \definecolor{ansi-green}{HTML}{00A250}
    \definecolor{ansi-green-intense}{HTML}{007427}
    \definecolor{ansi-yellow}{HTML}{DDB62B}
    \definecolor{ansi-yellow-intense}{HTML}{B27D12}
    \definecolor{ansi-blue}{HTML}{208FFB}
    \definecolor{ansi-blue-intense}{HTML}{0065CA}
    \definecolor{ansi-magenta}{HTML}{D160C4}
    \definecolor{ansi-magenta-intense}{HTML}{A03196}
    \definecolor{ansi-cyan}{HTML}{60C6C8}
    \definecolor{ansi-cyan-intense}{HTML}{258F8F}
    \definecolor{ansi-white}{HTML}{C5C1B4}
    \definecolor{ansi-white-intense}{HTML}{A1A6B2}
    \definecolor{ansi-default-inverse-fg}{HTML}{FFFFFF}
    \definecolor{ansi-default-inverse-bg}{HTML}{000000}

    % commands and environments needed by pandoc snippets
    % extracted from the output of `pandoc -s`
    \providecommand{\tightlist}{%
      \setlength{\itemsep}{0pt}\setlength{\parskip}{0pt}}
    \DefineVerbatimEnvironment{Highlighting}{Verbatim}{commandchars=\\\{\}}
    % Add ',fontsize=\small' for more characters per line
    \newenvironment{Shaded}{}{}
    \newcommand{\KeywordTok}[1]{\textcolor[rgb]{0.00,0.44,0.13}{\textbf{{#1}}}}
    \newcommand{\DataTypeTok}[1]{\textcolor[rgb]{0.56,0.13,0.00}{{#1}}}
    \newcommand{\DecValTok}[1]{\textcolor[rgb]{0.25,0.63,0.44}{{#1}}}
    \newcommand{\BaseNTok}[1]{\textcolor[rgb]{0.25,0.63,0.44}{{#1}}}
    \newcommand{\FloatTok}[1]{\textcolor[rgb]{0.25,0.63,0.44}{{#1}}}
    \newcommand{\CharTok}[1]{\textcolor[rgb]{0.25,0.44,0.63}{{#1}}}
    \newcommand{\StringTok}[1]{\textcolor[rgb]{0.25,0.44,0.63}{{#1}}}
    \newcommand{\CommentTok}[1]{\textcolor[rgb]{0.38,0.63,0.69}{\textit{{#1}}}}
    \newcommand{\OtherTok}[1]{\textcolor[rgb]{0.00,0.44,0.13}{{#1}}}
    \newcommand{\AlertTok}[1]{\textcolor[rgb]{1.00,0.00,0.00}{\textbf{{#1}}}}
    \newcommand{\FunctionTok}[1]{\textcolor[rgb]{0.02,0.16,0.49}{{#1}}}
    \newcommand{\RegionMarkerTok}[1]{{#1}}
    \newcommand{\ErrorTok}[1]{\textcolor[rgb]{1.00,0.00,0.00}{\textbf{{#1}}}}
    \newcommand{\NormalTok}[1]{{#1}}
    
    % Additional commands for more recent versions of Pandoc
    \newcommand{\ConstantTok}[1]{\textcolor[rgb]{0.53,0.00,0.00}{{#1}}}
    \newcommand{\SpecialCharTok}[1]{\textcolor[rgb]{0.25,0.44,0.63}{{#1}}}
    \newcommand{\VerbatimStringTok}[1]{\textcolor[rgb]{0.25,0.44,0.63}{{#1}}}
    \newcommand{\SpecialStringTok}[1]{\textcolor[rgb]{0.73,0.40,0.53}{{#1}}}
    \newcommand{\ImportTok}[1]{{#1}}
    \newcommand{\DocumentationTok}[1]{\textcolor[rgb]{0.73,0.13,0.13}{\textit{{#1}}}}
    \newcommand{\AnnotationTok}[1]{\textcolor[rgb]{0.38,0.63,0.69}{\textbf{\textit{{#1}}}}}
    \newcommand{\CommentVarTok}[1]{\textcolor[rgb]{0.38,0.63,0.69}{\textbf{\textit{{#1}}}}}
    \newcommand{\VariableTok}[1]{\textcolor[rgb]{0.10,0.09,0.49}{{#1}}}
    \newcommand{\ControlFlowTok}[1]{\textcolor[rgb]{0.00,0.44,0.13}{\textbf{{#1}}}}
    \newcommand{\OperatorTok}[1]{\textcolor[rgb]{0.40,0.40,0.40}{{#1}}}
    \newcommand{\BuiltInTok}[1]{{#1}}
    \newcommand{\ExtensionTok}[1]{{#1}}
    \newcommand{\PreprocessorTok}[1]{\textcolor[rgb]{0.74,0.48,0.00}{{#1}}}
    \newcommand{\AttributeTok}[1]{\textcolor[rgb]{0.49,0.56,0.16}{{#1}}}
    \newcommand{\InformationTok}[1]{\textcolor[rgb]{0.38,0.63,0.69}{\textbf{\textit{{#1}}}}}
    \newcommand{\WarningTok}[1]{\textcolor[rgb]{0.38,0.63,0.69}{\textbf{\textit{{#1}}}}}
    
    
    % Define a nice break command that doesn't care if a line doesn't already
    % exist.
    \def\br{\hspace*{\fill} \\* }
    % Math Jax compatibility definitions
    \def\gt{>}
    \def\lt{<}
    \let\Oldtex\TeX
    \let\Oldlatex\LaTeX
    \renewcommand{\TeX}{\textrm{\Oldtex}}
    \renewcommand{\LaTeX}{\textrm{\Oldlatex}}
    % Document parameters
    % Document title
    \title{Weather Situational Analysis}
    
    
    
    
    
% Pygments definitions
\makeatletter
\def\PY@reset{\let\PY@it=\relax \let\PY@bf=\relax%
    \let\PY@ul=\relax \let\PY@tc=\relax%
    \let\PY@bc=\relax \let\PY@ff=\relax}
\def\PY@tok#1{\csname PY@tok@#1\endcsname}
\def\PY@toks#1+{\ifx\relax#1\empty\else%
    \PY@tok{#1}\expandafter\PY@toks\fi}
\def\PY@do#1{\PY@bc{\PY@tc{\PY@ul{%
    \PY@it{\PY@bf{\PY@ff{#1}}}}}}}
\def\PY#1#2{\PY@reset\PY@toks#1+\relax+\PY@do{#2}}

\expandafter\def\csname PY@tok@w\endcsname{\def\PY@tc##1{\textcolor[rgb]{0.73,0.73,0.73}{##1}}}
\expandafter\def\csname PY@tok@c\endcsname{\let\PY@it=\textit\def\PY@tc##1{\textcolor[rgb]{0.25,0.50,0.50}{##1}}}
\expandafter\def\csname PY@tok@cp\endcsname{\def\PY@tc##1{\textcolor[rgb]{0.74,0.48,0.00}{##1}}}
\expandafter\def\csname PY@tok@k\endcsname{\let\PY@bf=\textbf\def\PY@tc##1{\textcolor[rgb]{0.00,0.50,0.00}{##1}}}
\expandafter\def\csname PY@tok@kp\endcsname{\def\PY@tc##1{\textcolor[rgb]{0.00,0.50,0.00}{##1}}}
\expandafter\def\csname PY@tok@kt\endcsname{\def\PY@tc##1{\textcolor[rgb]{0.69,0.00,0.25}{##1}}}
\expandafter\def\csname PY@tok@o\endcsname{\def\PY@tc##1{\textcolor[rgb]{0.40,0.40,0.40}{##1}}}
\expandafter\def\csname PY@tok@ow\endcsname{\let\PY@bf=\textbf\def\PY@tc##1{\textcolor[rgb]{0.67,0.13,1.00}{##1}}}
\expandafter\def\csname PY@tok@nb\endcsname{\def\PY@tc##1{\textcolor[rgb]{0.00,0.50,0.00}{##1}}}
\expandafter\def\csname PY@tok@nf\endcsname{\def\PY@tc##1{\textcolor[rgb]{0.00,0.00,1.00}{##1}}}
\expandafter\def\csname PY@tok@nc\endcsname{\let\PY@bf=\textbf\def\PY@tc##1{\textcolor[rgb]{0.00,0.00,1.00}{##1}}}
\expandafter\def\csname PY@tok@nn\endcsname{\let\PY@bf=\textbf\def\PY@tc##1{\textcolor[rgb]{0.00,0.00,1.00}{##1}}}
\expandafter\def\csname PY@tok@ne\endcsname{\let\PY@bf=\textbf\def\PY@tc##1{\textcolor[rgb]{0.82,0.25,0.23}{##1}}}
\expandafter\def\csname PY@tok@nv\endcsname{\def\PY@tc##1{\textcolor[rgb]{0.10,0.09,0.49}{##1}}}
\expandafter\def\csname PY@tok@no\endcsname{\def\PY@tc##1{\textcolor[rgb]{0.53,0.00,0.00}{##1}}}
\expandafter\def\csname PY@tok@nl\endcsname{\def\PY@tc##1{\textcolor[rgb]{0.63,0.63,0.00}{##1}}}
\expandafter\def\csname PY@tok@ni\endcsname{\let\PY@bf=\textbf\def\PY@tc##1{\textcolor[rgb]{0.60,0.60,0.60}{##1}}}
\expandafter\def\csname PY@tok@na\endcsname{\def\PY@tc##1{\textcolor[rgb]{0.49,0.56,0.16}{##1}}}
\expandafter\def\csname PY@tok@nt\endcsname{\let\PY@bf=\textbf\def\PY@tc##1{\textcolor[rgb]{0.00,0.50,0.00}{##1}}}
\expandafter\def\csname PY@tok@nd\endcsname{\def\PY@tc##1{\textcolor[rgb]{0.67,0.13,1.00}{##1}}}
\expandafter\def\csname PY@tok@s\endcsname{\def\PY@tc##1{\textcolor[rgb]{0.73,0.13,0.13}{##1}}}
\expandafter\def\csname PY@tok@sd\endcsname{\let\PY@it=\textit\def\PY@tc##1{\textcolor[rgb]{0.73,0.13,0.13}{##1}}}
\expandafter\def\csname PY@tok@si\endcsname{\let\PY@bf=\textbf\def\PY@tc##1{\textcolor[rgb]{0.73,0.40,0.53}{##1}}}
\expandafter\def\csname PY@tok@se\endcsname{\let\PY@bf=\textbf\def\PY@tc##1{\textcolor[rgb]{0.73,0.40,0.13}{##1}}}
\expandafter\def\csname PY@tok@sr\endcsname{\def\PY@tc##1{\textcolor[rgb]{0.73,0.40,0.53}{##1}}}
\expandafter\def\csname PY@tok@ss\endcsname{\def\PY@tc##1{\textcolor[rgb]{0.10,0.09,0.49}{##1}}}
\expandafter\def\csname PY@tok@sx\endcsname{\def\PY@tc##1{\textcolor[rgb]{0.00,0.50,0.00}{##1}}}
\expandafter\def\csname PY@tok@m\endcsname{\def\PY@tc##1{\textcolor[rgb]{0.40,0.40,0.40}{##1}}}
\expandafter\def\csname PY@tok@gh\endcsname{\let\PY@bf=\textbf\def\PY@tc##1{\textcolor[rgb]{0.00,0.00,0.50}{##1}}}
\expandafter\def\csname PY@tok@gu\endcsname{\let\PY@bf=\textbf\def\PY@tc##1{\textcolor[rgb]{0.50,0.00,0.50}{##1}}}
\expandafter\def\csname PY@tok@gd\endcsname{\def\PY@tc##1{\textcolor[rgb]{0.63,0.00,0.00}{##1}}}
\expandafter\def\csname PY@tok@gi\endcsname{\def\PY@tc##1{\textcolor[rgb]{0.00,0.63,0.00}{##1}}}
\expandafter\def\csname PY@tok@gr\endcsname{\def\PY@tc##1{\textcolor[rgb]{1.00,0.00,0.00}{##1}}}
\expandafter\def\csname PY@tok@ge\endcsname{\let\PY@it=\textit}
\expandafter\def\csname PY@tok@gs\endcsname{\let\PY@bf=\textbf}
\expandafter\def\csname PY@tok@gp\endcsname{\let\PY@bf=\textbf\def\PY@tc##1{\textcolor[rgb]{0.00,0.00,0.50}{##1}}}
\expandafter\def\csname PY@tok@go\endcsname{\def\PY@tc##1{\textcolor[rgb]{0.53,0.53,0.53}{##1}}}
\expandafter\def\csname PY@tok@gt\endcsname{\def\PY@tc##1{\textcolor[rgb]{0.00,0.27,0.87}{##1}}}
\expandafter\def\csname PY@tok@err\endcsname{\def\PY@bc##1{\setlength{\fboxsep}{0pt}\fcolorbox[rgb]{1.00,0.00,0.00}{1,1,1}{\strut ##1}}}
\expandafter\def\csname PY@tok@kc\endcsname{\let\PY@bf=\textbf\def\PY@tc##1{\textcolor[rgb]{0.00,0.50,0.00}{##1}}}
\expandafter\def\csname PY@tok@kd\endcsname{\let\PY@bf=\textbf\def\PY@tc##1{\textcolor[rgb]{0.00,0.50,0.00}{##1}}}
\expandafter\def\csname PY@tok@kn\endcsname{\let\PY@bf=\textbf\def\PY@tc##1{\textcolor[rgb]{0.00,0.50,0.00}{##1}}}
\expandafter\def\csname PY@tok@kr\endcsname{\let\PY@bf=\textbf\def\PY@tc##1{\textcolor[rgb]{0.00,0.50,0.00}{##1}}}
\expandafter\def\csname PY@tok@bp\endcsname{\def\PY@tc##1{\textcolor[rgb]{0.00,0.50,0.00}{##1}}}
\expandafter\def\csname PY@tok@fm\endcsname{\def\PY@tc##1{\textcolor[rgb]{0.00,0.00,1.00}{##1}}}
\expandafter\def\csname PY@tok@vc\endcsname{\def\PY@tc##1{\textcolor[rgb]{0.10,0.09,0.49}{##1}}}
\expandafter\def\csname PY@tok@vg\endcsname{\def\PY@tc##1{\textcolor[rgb]{0.10,0.09,0.49}{##1}}}
\expandafter\def\csname PY@tok@vi\endcsname{\def\PY@tc##1{\textcolor[rgb]{0.10,0.09,0.49}{##1}}}
\expandafter\def\csname PY@tok@vm\endcsname{\def\PY@tc##1{\textcolor[rgb]{0.10,0.09,0.49}{##1}}}
\expandafter\def\csname PY@tok@sa\endcsname{\def\PY@tc##1{\textcolor[rgb]{0.73,0.13,0.13}{##1}}}
\expandafter\def\csname PY@tok@sb\endcsname{\def\PY@tc##1{\textcolor[rgb]{0.73,0.13,0.13}{##1}}}
\expandafter\def\csname PY@tok@sc\endcsname{\def\PY@tc##1{\textcolor[rgb]{0.73,0.13,0.13}{##1}}}
\expandafter\def\csname PY@tok@dl\endcsname{\def\PY@tc##1{\textcolor[rgb]{0.73,0.13,0.13}{##1}}}
\expandafter\def\csname PY@tok@s2\endcsname{\def\PY@tc##1{\textcolor[rgb]{0.73,0.13,0.13}{##1}}}
\expandafter\def\csname PY@tok@sh\endcsname{\def\PY@tc##1{\textcolor[rgb]{0.73,0.13,0.13}{##1}}}
\expandafter\def\csname PY@tok@s1\endcsname{\def\PY@tc##1{\textcolor[rgb]{0.73,0.13,0.13}{##1}}}
\expandafter\def\csname PY@tok@mb\endcsname{\def\PY@tc##1{\textcolor[rgb]{0.40,0.40,0.40}{##1}}}
\expandafter\def\csname PY@tok@mf\endcsname{\def\PY@tc##1{\textcolor[rgb]{0.40,0.40,0.40}{##1}}}
\expandafter\def\csname PY@tok@mh\endcsname{\def\PY@tc##1{\textcolor[rgb]{0.40,0.40,0.40}{##1}}}
\expandafter\def\csname PY@tok@mi\endcsname{\def\PY@tc##1{\textcolor[rgb]{0.40,0.40,0.40}{##1}}}
\expandafter\def\csname PY@tok@il\endcsname{\def\PY@tc##1{\textcolor[rgb]{0.40,0.40,0.40}{##1}}}
\expandafter\def\csname PY@tok@mo\endcsname{\def\PY@tc##1{\textcolor[rgb]{0.40,0.40,0.40}{##1}}}
\expandafter\def\csname PY@tok@ch\endcsname{\let\PY@it=\textit\def\PY@tc##1{\textcolor[rgb]{0.25,0.50,0.50}{##1}}}
\expandafter\def\csname PY@tok@cm\endcsname{\let\PY@it=\textit\def\PY@tc##1{\textcolor[rgb]{0.25,0.50,0.50}{##1}}}
\expandafter\def\csname PY@tok@cpf\endcsname{\let\PY@it=\textit\def\PY@tc##1{\textcolor[rgb]{0.25,0.50,0.50}{##1}}}
\expandafter\def\csname PY@tok@c1\endcsname{\let\PY@it=\textit\def\PY@tc##1{\textcolor[rgb]{0.25,0.50,0.50}{##1}}}
\expandafter\def\csname PY@tok@cs\endcsname{\let\PY@it=\textit\def\PY@tc##1{\textcolor[rgb]{0.25,0.50,0.50}{##1}}}

\def\PYZbs{\char`\\}
\def\PYZus{\char`\_}
\def\PYZob{\char`\{}
\def\PYZcb{\char`\}}
\def\PYZca{\char`\^}
\def\PYZam{\char`\&}
\def\PYZlt{\char`\<}
\def\PYZgt{\char`\>}
\def\PYZsh{\char`\#}
\def\PYZpc{\char`\%}
\def\PYZdl{\char`\$}
\def\PYZhy{\char`\-}
\def\PYZsq{\char`\'}
\def\PYZdq{\char`\"}
\def\PYZti{\char`\~}
% for compatibility with earlier versions
\def\PYZat{@}
\def\PYZlb{[}
\def\PYZrb{]}
\makeatother


    % For linebreaks inside Verbatim environment from package fancyvrb. 
    \makeatletter
        \newbox\Wrappedcontinuationbox 
        \newbox\Wrappedvisiblespacebox 
        \newcommand*\Wrappedvisiblespace {\textcolor{red}{\textvisiblespace}} 
        \newcommand*\Wrappedcontinuationsymbol {\textcolor{red}{\llap{\tiny$\m@th\hookrightarrow$}}} 
        \newcommand*\Wrappedcontinuationindent {3ex } 
        \newcommand*\Wrappedafterbreak {\kern\Wrappedcontinuationindent\copy\Wrappedcontinuationbox} 
        % Take advantage of the already applied Pygments mark-up to insert 
        % potential linebreaks for TeX processing. 
        %        {, <, #, %, $, ' and ": go to next line. 
        %        _, }, ^, &, >, - and ~: stay at end of broken line. 
        % Use of \textquotesingle for straight quote. 
        \newcommand*\Wrappedbreaksatspecials {% 
            \def\PYGZus{\discretionary{\char`\_}{\Wrappedafterbreak}{\char`\_}}% 
            \def\PYGZob{\discretionary{}{\Wrappedafterbreak\char`\{}{\char`\{}}% 
            \def\PYGZcb{\discretionary{\char`\}}{\Wrappedafterbreak}{\char`\}}}% 
            \def\PYGZca{\discretionary{\char`\^}{\Wrappedafterbreak}{\char`\^}}% 
            \def\PYGZam{\discretionary{\char`\&}{\Wrappedafterbreak}{\char`\&}}% 
            \def\PYGZlt{\discretionary{}{\Wrappedafterbreak\char`\<}{\char`\<}}% 
            \def\PYGZgt{\discretionary{\char`\>}{\Wrappedafterbreak}{\char`\>}}% 
            \def\PYGZsh{\discretionary{}{\Wrappedafterbreak\char`\#}{\char`\#}}% 
            \def\PYGZpc{\discretionary{}{\Wrappedafterbreak\char`\%}{\char`\%}}% 
            \def\PYGZdl{\discretionary{}{\Wrappedafterbreak\char`\$}{\char`\$}}% 
            \def\PYGZhy{\discretionary{\char`\-}{\Wrappedafterbreak}{\char`\-}}% 
            \def\PYGZsq{\discretionary{}{\Wrappedafterbreak\textquotesingle}{\textquotesingle}}% 
            \def\PYGZdq{\discretionary{}{\Wrappedafterbreak\char`\"}{\char`\"}}% 
            \def\PYGZti{\discretionary{\char`\~}{\Wrappedafterbreak}{\char`\~}}% 
        } 
        % Some characters . , ; ? ! / are not pygmentized. 
        % This macro makes them "active" and they will insert potential linebreaks 
        \newcommand*\Wrappedbreaksatpunct {% 
            \lccode`\~`\.\lowercase{\def~}{\discretionary{\hbox{\char`\.}}{\Wrappedafterbreak}{\hbox{\char`\.}}}% 
            \lccode`\~`\,\lowercase{\def~}{\discretionary{\hbox{\char`\,}}{\Wrappedafterbreak}{\hbox{\char`\,}}}% 
            \lccode`\~`\;\lowercase{\def~}{\discretionary{\hbox{\char`\;}}{\Wrappedafterbreak}{\hbox{\char`\;}}}% 
            \lccode`\~`\:\lowercase{\def~}{\discretionary{\hbox{\char`\:}}{\Wrappedafterbreak}{\hbox{\char`\:}}}% 
            \lccode`\~`\?\lowercase{\def~}{\discretionary{\hbox{\char`\?}}{\Wrappedafterbreak}{\hbox{\char`\?}}}% 
            \lccode`\~`\!\lowercase{\def~}{\discretionary{\hbox{\char`\!}}{\Wrappedafterbreak}{\hbox{\char`\!}}}% 
            \lccode`\~`\/\lowercase{\def~}{\discretionary{\hbox{\char`\/}}{\Wrappedafterbreak}{\hbox{\char`\/}}}% 
            \catcode`\.\active
            \catcode`\,\active 
            \catcode`\;\active
            \catcode`\:\active
            \catcode`\?\active
            \catcode`\!\active
            \catcode`\/\active 
            \lccode`\~`\~ 	
        }
    \makeatother

    \let\OriginalVerbatim=\Verbatim
    \makeatletter
    \renewcommand{\Verbatim}[1][1]{%
        %\parskip\z@skip
        \sbox\Wrappedcontinuationbox {\Wrappedcontinuationsymbol}%
        \sbox\Wrappedvisiblespacebox {\FV@SetupFont\Wrappedvisiblespace}%
        \def\FancyVerbFormatLine ##1{\hsize\linewidth
            \vtop{\raggedright\hyphenpenalty\z@\exhyphenpenalty\z@
                \doublehyphendemerits\z@\finalhyphendemerits\z@
                \strut ##1\strut}%
        }%
        % If the linebreak is at a space, the latter will be displayed as visible
        % space at end of first line, and a continuation symbol starts next line.
        % Stretch/shrink are however usually zero for typewriter font.
        \def\FV@Space {%
            \nobreak\hskip\z@ plus\fontdimen3\font minus\fontdimen4\font
            \discretionary{\copy\Wrappedvisiblespacebox}{\Wrappedafterbreak}
            {\kern\fontdimen2\font}%
        }%
        
        % Allow breaks at special characters using \PYG... macros.
        \Wrappedbreaksatspecials
        % Breaks at punctuation characters . , ; ? ! and / need catcode=\active 	
        \OriginalVerbatim[#1,codes*=\Wrappedbreaksatpunct]%
    }
    \makeatother

    % Exact colors from NB
    \definecolor{incolor}{HTML}{303F9F}
    \definecolor{outcolor}{HTML}{D84315}
    \definecolor{cellborder}{HTML}{CFCFCF}
    \definecolor{cellbackground}{HTML}{F7F7F7}
    
    % prompt
    \makeatletter
    \newcommand{\boxspacing}{\kern\kvtcb@left@rule\kern\kvtcb@boxsep}
    \makeatother
    \newcommand{\prompt}[4]{
        \ttfamily\llap{{\color{#2}[#3]:\hspace{3pt}#4}}\vspace{-\baselineskip}
    }
    

    
    % Prevent overflowing lines due to hard-to-break entities
    \sloppy 
    % Setup hyperref package
    \hypersetup{
      breaklinks=true,  % so long urls are correctly broken across lines
      colorlinks=true,
      urlcolor=urlcolor,
      linkcolor=linkcolor,
      citecolor=citecolor,
      }
    % Slightly bigger margins than the latex defaults
    
    \geometry{verbose,tmargin=1in,bmargin=1in,lmargin=1in,rmargin=1in}
    
    

\begin{document}
    
    \maketitle
    
    

    
    \begin{tcolorbox}[breakable, size=fbox, boxrule=1pt, pad at break*=1mm,colback=cellbackground, colframe=cellborder]
\prompt{In}{incolor}{1}{\boxspacing}
\begin{Verbatim}[commandchars=\\\{\}]
\PY{k+kn}{import} \PY{n+nn}{numpy} \PY{k}{as} \PY{n+nn}{np} \PY{c+c1}{\PYZsh{}\PYZsh{} For Linear Algebra}
\PY{k+kn}{import} \PY{n+nn}{pandas} \PY{k}{as} \PY{n+nn}{pd} \PY{c+c1}{\PYZsh{}\PYZsh{} To Work With Data}
\PY{c+c1}{\PYZsh{}\PYZsh{} For visualizations I\PYZsq{}ll be using plotly package, this creates interesting and interective visualizations.}
\PY{k+kn}{import} \PY{n+nn}{plotly}\PY{n+nn}{.}\PY{n+nn}{express} \PY{k}{as} \PY{n+nn}{px} 
\PY{k+kn}{import} \PY{n+nn}{plotly}\PY{n+nn}{.}\PY{n+nn}{graph\PYZus{}objects} \PY{k}{as} \PY{n+nn}{go}
\PY{k+kn}{from} \PY{n+nn}{plotly}\PY{n+nn}{.}\PY{n+nn}{subplots} \PY{k}{import} \PY{n}{make\PYZus{}subplots}
\PY{k+kn}{from} \PY{n+nn}{datetime} \PY{k}{import} \PY{n}{datetime} \PY{c+c1}{\PYZsh{}\PYZsh{} Time Series analysis.}
\end{Verbatim}
\end{tcolorbox}

    \begin{tcolorbox}[breakable, size=fbox, boxrule=1pt, pad at break*=1mm,colback=cellbackground, colframe=cellborder]
\prompt{In}{incolor}{2}{\boxspacing}
\begin{Verbatim}[commandchars=\\\{\}]
\PY{n}{df} \PY{o}{=} \PY{n}{pd}\PY{o}{.}\PY{n}{read\PYZus{}csv}\PY{p}{(}\PY{l+s+sa}{r}\PY{l+s+s1}{\PYZsq{}}\PY{l+s+s1}{C:}\PY{l+s+s1}{\PYZbs{}}\PY{l+s+s1}{Users}\PY{l+s+s1}{\PYZbs{}}\PY{l+s+s1}{shubham.kj}\PY{l+s+s1}{\PYZbs{}}\PY{l+s+s1}{Desktop}\PY{l+s+s1}{\PYZbs{}}\PY{l+s+s1}{weatherIND.csv}\PY{l+s+s1}{\PYZsq{}}\PY{p}{)}
\end{Verbatim}
\end{tcolorbox}

    \begin{tcolorbox}[breakable, size=fbox, boxrule=1pt, pad at break*=1mm,colback=cellbackground, colframe=cellborder]
\prompt{In}{incolor}{3}{\boxspacing}
\begin{Verbatim}[commandchars=\\\{\}]
\PY{n}{df}\PY{o}{.}\PY{n}{head}\PY{p}{(}\PY{p}{)}
\end{Verbatim}
\end{tcolorbox}

            \begin{tcolorbox}[breakable, size=fbox, boxrule=.5pt, pad at break*=1mm, opacityfill=0]
\prompt{Out}{outcolor}{3}{\boxspacing}
\begin{Verbatim}[commandchars=\\\{\}]
   Unnamed: 0  YEAR    JAN    FEB    MAR    APR    MAY    JUN    JUL    AUG  \textbackslash{}
0           0  1901  17.99  19.43  23.49  26.41  28.28  28.60  27.49  26.98
1           1  1902  19.00  20.39  24.10  26.54  28.68  28.44  27.29  27.05
2           2  1903  18.32  19.79  22.46  26.03  27.93  28.41  28.04  26.63
3           3  1904  17.77  19.39  22.95  26.73  27.83  27.85  26.84  26.73
4           4  1905  17.40  17.79  21.78  24.84  28.32  28.69  27.67  27.47

     SEP    OCT    NOV    DEC
0  26.26  25.08  21.73  18.95
1  25.95  24.37  21.33  18.78
2  26.34  24.57  20.96  18.29
3  25.84  24.36  21.07  18.84
4  26.29  26.16  22.07  18.71
\end{Verbatim}
\end{tcolorbox}
        
    \begin{tcolorbox}[breakable, size=fbox, boxrule=1pt, pad at break*=1mm,colback=cellbackground, colframe=cellborder]
\prompt{In}{incolor}{4}{\boxspacing}
\begin{Verbatim}[commandchars=\\\{\}]
\PY{n}{df} \PY{o}{=} \PY{n}{pd}\PY{o}{.}\PY{n}{read\PYZus{}csv}\PY{p}{(}\PY{l+s+sa}{r}\PY{l+s+s1}{\PYZsq{}}\PY{l+s+s1}{C:}\PY{l+s+s1}{\PYZbs{}}\PY{l+s+s1}{Users}\PY{l+s+s1}{\PYZbs{}}\PY{l+s+s1}{shubham.kj}\PY{l+s+s1}{\PYZbs{}}\PY{l+s+s1}{Desktop}\PY{l+s+s1}{\PYZbs{}}\PY{l+s+s1}{weatherIND.csv}\PY{l+s+s1}{\PYZsq{}}\PY{p}{,} \PY{n}{index\PYZus{}col}\PY{o}{=}\PY{l+m+mi}{0}\PY{p}{)}
\end{Verbatim}
\end{tcolorbox}

    \begin{tcolorbox}[breakable, size=fbox, boxrule=1pt, pad at break*=1mm,colback=cellbackground, colframe=cellborder]
\prompt{In}{incolor}{5}{\boxspacing}
\begin{Verbatim}[commandchars=\\\{\}]
\PY{n}{df}\PY{o}{.}\PY{n}{head}\PY{p}{(}\PY{p}{)} 
\end{Verbatim}
\end{tcolorbox}

            \begin{tcolorbox}[breakable, size=fbox, boxrule=.5pt, pad at break*=1mm, opacityfill=0]
\prompt{Out}{outcolor}{5}{\boxspacing}
\begin{Verbatim}[commandchars=\\\{\}]
   YEAR    JAN    FEB    MAR    APR    MAY    JUN    JUL    AUG    SEP    OCT  \textbackslash{}
0  1901  17.99  19.43  23.49  26.41  28.28  28.60  27.49  26.98  26.26  25.08
1  1902  19.00  20.39  24.10  26.54  28.68  28.44  27.29  27.05  25.95  24.37
2  1903  18.32  19.79  22.46  26.03  27.93  28.41  28.04  26.63  26.34  24.57
3  1904  17.77  19.39  22.95  26.73  27.83  27.85  26.84  26.73  25.84  24.36
4  1905  17.40  17.79  21.78  24.84  28.32  28.69  27.67  27.47  26.29  26.16

     NOV    DEC
0  21.73  18.95
1  21.33  18.78
2  20.96  18.29
3  21.07  18.84
4  22.07  18.71
\end{Verbatim}
\end{tcolorbox}
        
    \hypertarget{data-melting}{%
\section{Data Melting}\label{data-melting}}

    \begin{tcolorbox}[breakable, size=fbox, boxrule=1pt, pad at break*=1mm,colback=cellbackground, colframe=cellborder]
\prompt{In}{incolor}{6}{\boxspacing}
\begin{Verbatim}[commandchars=\\\{\}]
\PY{n}{df1} \PY{o}{=} \PY{n}{pd}\PY{o}{.}\PY{n}{melt}\PY{p}{(}\PY{n}{df}\PY{p}{,} \PY{n}{id\PYZus{}vars}\PY{o}{=}\PY{l+s+s1}{\PYZsq{}}\PY{l+s+s1}{YEAR}\PY{l+s+s1}{\PYZsq{}}\PY{p}{,} \PY{n}{value\PYZus{}vars}\PY{o}{=}\PY{n}{df}\PY{o}{.}\PY{n}{columns}\PY{p}{[}\PY{l+m+mi}{1}\PY{p}{:}\PY{p}{]}\PY{p}{)} \PY{c+c1}{\PYZsh{}\PYZsh{} This will melt the data}
\PY{n}{df1}\PY{o}{.}\PY{n}{head}\PY{p}{(}\PY{p}{)} \PY{c+c1}{\PYZsh{}\PYZsh{} This is how the new data looks now:}
\end{Verbatim}
\end{tcolorbox}

            \begin{tcolorbox}[breakable, size=fbox, boxrule=.5pt, pad at break*=1mm, opacityfill=0]
\prompt{Out}{outcolor}{6}{\boxspacing}
\begin{Verbatim}[commandchars=\\\{\}]
   YEAR variable  value
0  1901      JAN  17.99
1  1902      JAN  19.00
2  1903      JAN  18.32
3  1904      JAN  17.77
4  1905      JAN  17.40
\end{Verbatim}
\end{tcolorbox}
        
    \begin{tcolorbox}[breakable, size=fbox, boxrule=1pt, pad at break*=1mm,colback=cellbackground, colframe=cellborder]
\prompt{In}{incolor}{7}{\boxspacing}
\begin{Verbatim}[commandchars=\\\{\}]
\PY{n}{df1}\PY{p}{[}\PY{l+s+s1}{\PYZsq{}}\PY{l+s+s1}{Date}\PY{l+s+s1}{\PYZsq{}}\PY{p}{]} \PY{o}{=} \PY{n}{df1}\PY{p}{[}\PY{l+s+s1}{\PYZsq{}}\PY{l+s+s1}{variable}\PY{l+s+s1}{\PYZsq{}}\PY{p}{]} \PY{o}{+} \PY{l+s+s1}{\PYZsq{}}\PY{l+s+s1}{ }\PY{l+s+s1}{\PYZsq{}} \PY{o}{+} \PY{n}{df1}\PY{p}{[}\PY{l+s+s1}{\PYZsq{}}\PY{l+s+s1}{YEAR}\PY{l+s+s1}{\PYZsq{}}\PY{p}{]}\PY{o}{.}\PY{n}{astype}\PY{p}{(}\PY{n+nb}{str}\PY{p}{)}  
\PY{n}{df1}\PY{o}{.}\PY{n}{loc}\PY{p}{[}\PY{p}{:}\PY{p}{,}\PY{l+s+s1}{\PYZsq{}}\PY{l+s+s1}{Date}\PY{l+s+s1}{\PYZsq{}}\PY{p}{]} \PY{o}{=} \PY{n}{df1}\PY{p}{[}\PY{l+s+s1}{\PYZsq{}}\PY{l+s+s1}{Date}\PY{l+s+s1}{\PYZsq{}}\PY{p}{]}\PY{o}{.}\PY{n}{apply}\PY{p}{(}\PY{k}{lambda} \PY{n}{x} \PY{p}{:} \PY{n}{datetime}\PY{o}{.}\PY{n}{strptime}\PY{p}{(}\PY{n}{x}\PY{p}{,} \PY{l+s+s1}{\PYZsq{}}\PY{l+s+s1}{\PYZpc{}}\PY{l+s+s1}{b }\PY{l+s+s1}{\PYZpc{}}\PY{l+s+s1}{Y}\PY{l+s+s1}{\PYZsq{}}\PY{p}{)}\PY{p}{)} \PY{c+c1}{\PYZsh{}\PYZsh{} Converting String to datetime object}
\PY{n}{df1}\PY{o}{.}\PY{n}{head}\PY{p}{(}\PY{p}{)}
\end{Verbatim}
\end{tcolorbox}

            \begin{tcolorbox}[breakable, size=fbox, boxrule=.5pt, pad at break*=1mm, opacityfill=0]
\prompt{Out}{outcolor}{7}{\boxspacing}
\begin{Verbatim}[commandchars=\\\{\}]
   YEAR variable  value       Date
0  1901      JAN  17.99 1901-01-01
1  1902      JAN  19.00 1902-01-01
2  1903      JAN  18.32 1903-01-01
3  1904      JAN  17.77 1904-01-01
4  1905      JAN  17.40 1905-01-01
\end{Verbatim}
\end{tcolorbox}
        
    \hypertarget{time-series-temperature-analysis}{%
\section{Time Series Temperature
Analysis}\label{time-series-temperature-analysis}}

    \begin{tcolorbox}[breakable, size=fbox, boxrule=1pt, pad at break*=1mm,colback=cellbackground, colframe=cellborder]
\prompt{In}{incolor}{8}{\boxspacing}
\begin{Verbatim}[commandchars=\\\{\}]
\PY{n}{df1}\PY{o}{.}\PY{n}{columns}\PY{o}{=}\PY{p}{[}\PY{l+s+s1}{\PYZsq{}}\PY{l+s+s1}{Year}\PY{l+s+s1}{\PYZsq{}}\PY{p}{,} \PY{l+s+s1}{\PYZsq{}}\PY{l+s+s1}{Month}\PY{l+s+s1}{\PYZsq{}}\PY{p}{,} \PY{l+s+s1}{\PYZsq{}}\PY{l+s+s1}{Temprature}\PY{l+s+s1}{\PYZsq{}}\PY{p}{,} \PY{l+s+s1}{\PYZsq{}}\PY{l+s+s1}{Date}\PY{l+s+s1}{\PYZsq{}}\PY{p}{]}
\PY{n}{df1}\PY{o}{.}\PY{n}{sort\PYZus{}values}\PY{p}{(}\PY{n}{by}\PY{o}{=}\PY{l+s+s1}{\PYZsq{}}\PY{l+s+s1}{Date}\PY{l+s+s1}{\PYZsq{}}\PY{p}{,} \PY{n}{inplace}\PY{o}{=}\PY{k+kc}{True}\PY{p}{)} \PY{c+c1}{\PYZsh{}\PYZsh{} To get the time series right.}
\PY{n}{fig} \PY{o}{=} \PY{n}{go}\PY{o}{.}\PY{n}{Figure}\PY{p}{(}\PY{n}{layout} \PY{o}{=} \PY{n}{go}\PY{o}{.}\PY{n}{Layout}\PY{p}{(}\PY{n}{yaxis}\PY{o}{=}\PY{n+nb}{dict}\PY{p}{(}\PY{n+nb}{range}\PY{o}{=}\PY{p}{[}\PY{l+m+mi}{0}\PY{p}{,} \PY{n}{df1}\PY{p}{[}\PY{l+s+s1}{\PYZsq{}}\PY{l+s+s1}{Temprature}\PY{l+s+s1}{\PYZsq{}}\PY{p}{]}\PY{o}{.}\PY{n}{max}\PY{p}{(}\PY{p}{)}\PY{o}{+}\PY{l+m+mi}{1}\PY{p}{]}\PY{p}{)}\PY{p}{)}\PY{p}{)}
\PY{n}{fig}\PY{o}{.}\PY{n}{add\PYZus{}trace}\PY{p}{(}\PY{n}{go}\PY{o}{.}\PY{n}{Scatter}\PY{p}{(}\PY{n}{x}\PY{o}{=}\PY{n}{df1}\PY{p}{[}\PY{l+s+s1}{\PYZsq{}}\PY{l+s+s1}{Date}\PY{l+s+s1}{\PYZsq{}}\PY{p}{]}\PY{p}{,} \PY{n}{y}\PY{o}{=}\PY{n}{df1}\PY{p}{[}\PY{l+s+s1}{\PYZsq{}}\PY{l+s+s1}{Temprature}\PY{l+s+s1}{\PYZsq{}}\PY{p}{]}\PY{p}{)}\PY{p}{,} \PY{p}{)}
\PY{n}{fig}\PY{o}{.}\PY{n}{update\PYZus{}layout}\PY{p}{(}\PY{n}{title}\PY{o}{=}\PY{l+s+s1}{\PYZsq{}}\PY{l+s+s1}{Temprature Throught Timeline:}\PY{l+s+s1}{\PYZsq{}}\PY{p}{,}
                 \PY{n}{xaxis\PYZus{}title}\PY{o}{=}\PY{l+s+s1}{\PYZsq{}}\PY{l+s+s1}{Time}\PY{l+s+s1}{\PYZsq{}}\PY{p}{,} \PY{n}{yaxis\PYZus{}title}\PY{o}{=}\PY{l+s+s1}{\PYZsq{}}\PY{l+s+s1}{Temprature in Degrees}\PY{l+s+s1}{\PYZsq{}}\PY{p}{)}
\PY{n}{fig}\PY{o}{.}\PY{n}{update\PYZus{}layout}\PY{p}{(}\PY{n}{xaxis}\PY{o}{=}\PY{n}{go}\PY{o}{.}\PY{n}{layout}\PY{o}{.}\PY{n}{XAxis}\PY{p}{(}
    \PY{n}{rangeselector}\PY{o}{=}\PY{n+nb}{dict}\PY{p}{(}
        \PY{n}{buttons}\PY{o}{=}\PY{n+nb}{list}\PY{p}{(}\PY{p}{[}\PY{n+nb}{dict}\PY{p}{(}\PY{n}{label}\PY{o}{=}\PY{l+s+s2}{\PYZdq{}}\PY{l+s+s2}{Whole View}\PY{l+s+s2}{\PYZdq{}}\PY{p}{,} \PY{n}{step}\PY{o}{=}\PY{l+s+s2}{\PYZdq{}}\PY{l+s+s2}{all}\PY{l+s+s2}{\PYZdq{}}\PY{p}{)}\PY{p}{,}
                      \PY{n+nb}{dict}\PY{p}{(}\PY{n}{count}\PY{o}{=}\PY{l+m+mi}{1}\PY{p}{,}\PY{n}{label}\PY{o}{=}\PY{l+s+s2}{\PYZdq{}}\PY{l+s+s2}{One Year View}\PY{l+s+s2}{\PYZdq{}}\PY{p}{,}\PY{n}{step}\PY{o}{=}\PY{l+s+s2}{\PYZdq{}}\PY{l+s+s2}{year}\PY{l+s+s2}{\PYZdq{}}\PY{p}{,}\PY{n}{stepmode}\PY{o}{=}\PY{l+s+s2}{\PYZdq{}}\PY{l+s+s2}{todate}\PY{l+s+s2}{\PYZdq{}}\PY{p}{)}                      
                     \PY{p}{]}\PY{p}{)}\PY{p}{)}\PY{p}{,}
        \PY{n}{rangeslider}\PY{o}{=}\PY{n+nb}{dict}\PY{p}{(}\PY{n}{visible}\PY{o}{=}\PY{k+kc}{True}\PY{p}{)}\PY{p}{,}\PY{n+nb}{type}\PY{o}{=}\PY{l+s+s2}{\PYZdq{}}\PY{l+s+s2}{date}\PY{l+s+s2}{\PYZdq{}}\PY{p}{)}
\PY{p}{)}
\PY{n}{fig}\PY{o}{.}\PY{n}{show}\PY{p}{(}\PY{p}{)}
\end{Verbatim}
\end{tcolorbox}

    
    
    
    
    Insights: May 1921 has been the hottest month in india in the history.
Dec, Jan and Feb are the coldest months. One could group them together
as ``Winter''. Apr, May, Jun, July and Aug are the hottest months. One
could group them together as ``Summer''. But, since this is not how
seasons work. We have four main seasons in India and this is how they
are grouped: Winter : December, January and February. Summer(Also
called, ``Pre Monsoon Season'') : March, April and May. Monsoon : June,
July, August and September. Autumn(Also called "Post Monsoon Season) :
October and November.

    \hypertarget{warmestcoldestaverage}{%
\section{Warmest/Coldest/Average}\label{warmestcoldestaverage}}

    \begin{tcolorbox}[breakable, size=fbox, boxrule=1pt, pad at break*=1mm,colback=cellbackground, colframe=cellborder]
\prompt{In}{incolor}{9}{\boxspacing}
\begin{Verbatim}[commandchars=\\\{\}]
\PY{n}{fig} \PY{o}{=} \PY{n}{px}\PY{o}{.}\PY{n}{box}\PY{p}{(}\PY{n}{df1}\PY{p}{,} \PY{l+s+s1}{\PYZsq{}}\PY{l+s+s1}{Month}\PY{l+s+s1}{\PYZsq{}}\PY{p}{,} \PY{l+s+s1}{\PYZsq{}}\PY{l+s+s1}{Temprature}\PY{l+s+s1}{\PYZsq{}}\PY{p}{)}
\PY{n}{fig}\PY{o}{.}\PY{n}{update\PYZus{}layout}\PY{p}{(}\PY{n}{title}\PY{o}{=}\PY{l+s+s1}{\PYZsq{}}\PY{l+s+s1}{Warmest, Coldest and Median Monthly Tempratue.}\PY{l+s+s1}{\PYZsq{}}\PY{p}{)}
\PY{n}{fig}\PY{o}{.}\PY{n}{show}\PY{p}{(}\PY{p}{)}
\end{Verbatim}
\end{tcolorbox}

    
    
    Insights: January has the coldest Days in an Year. May has the hottest
days in an Year. July is the month with least Standard Daviation which
means, temprature in july vary least. We can expect any day in july to
be a warm day.

    \begin{tcolorbox}[breakable, size=fbox, boxrule=1pt, pad at break*=1mm,colback=cellbackground, colframe=cellborder]
\prompt{In}{incolor}{10}{\boxspacing}
\begin{Verbatim}[commandchars=\\\{\}]
\PY{k+kn}{from} \PY{n+nn}{sklearn}\PY{n+nn}{.}\PY{n+nn}{cluster} \PY{k}{import} \PY{n}{KMeans}
\PY{n}{sse} \PY{o}{=} \PY{p}{[}\PY{p}{]}
\PY{n}{target} \PY{o}{=} \PY{n}{df1}\PY{p}{[}\PY{l+s+s1}{\PYZsq{}}\PY{l+s+s1}{Temprature}\PY{l+s+s1}{\PYZsq{}}\PY{p}{]}\PY{o}{.}\PY{n}{to\PYZus{}numpy}\PY{p}{(}\PY{p}{)}\PY{o}{.}\PY{n}{reshape}\PY{p}{(}\PY{o}{\PYZhy{}}\PY{l+m+mi}{1}\PY{p}{,}\PY{l+m+mi}{1}\PY{p}{)}
\PY{n}{num\PYZus{}clusters} \PY{o}{=} \PY{n+nb}{list}\PY{p}{(}\PY{n+nb}{range}\PY{p}{(}\PY{l+m+mi}{1}\PY{p}{,} \PY{l+m+mi}{10}\PY{p}{)}\PY{p}{)}

\PY{k}{for} \PY{n}{k} \PY{o+ow}{in} \PY{n}{num\PYZus{}clusters}\PY{p}{:}
    \PY{n}{km} \PY{o}{=} \PY{n}{KMeans}\PY{p}{(}\PY{n}{n\PYZus{}clusters}\PY{o}{=}\PY{n}{k}\PY{p}{)}
    \PY{n}{km}\PY{o}{.}\PY{n}{fit}\PY{p}{(}\PY{n}{target}\PY{p}{)}
    \PY{n}{sse}\PY{o}{.}\PY{n}{append}\PY{p}{(}\PY{n}{km}\PY{o}{.}\PY{n}{inertia\PYZus{}}\PY{p}{)}

\PY{n}{fig} \PY{o}{=} \PY{n}{go}\PY{o}{.}\PY{n}{Figure}\PY{p}{(}\PY{n}{data}\PY{o}{=}\PY{p}{[}
    \PY{n}{go}\PY{o}{.}\PY{n}{Scatter}\PY{p}{(}\PY{n}{x} \PY{o}{=} \PY{n}{num\PYZus{}clusters}\PY{p}{,} \PY{n}{y}\PY{o}{=}\PY{n}{sse}\PY{p}{,} \PY{n}{mode}\PY{o}{=}\PY{l+s+s1}{\PYZsq{}}\PY{l+s+s1}{lines}\PY{l+s+s1}{\PYZsq{}}\PY{p}{)}\PY{p}{,}
    \PY{n}{go}\PY{o}{.}\PY{n}{Scatter}\PY{p}{(}\PY{n}{x} \PY{o}{=} \PY{n}{num\PYZus{}clusters}\PY{p}{,} \PY{n}{y}\PY{o}{=}\PY{n}{sse}\PY{p}{,} \PY{n}{mode}\PY{o}{=}\PY{l+s+s1}{\PYZsq{}}\PY{l+s+s1}{markers}\PY{l+s+s1}{\PYZsq{}}\PY{p}{)}
\PY{p}{]}\PY{p}{)}

\PY{n}{fig}\PY{o}{.}\PY{n}{update\PYZus{}layout}\PY{p}{(}\PY{n}{title}\PY{o}{=}\PY{l+s+s2}{\PYZdq{}}\PY{l+s+s2}{Evaluation on number of clusters:}\PY{l+s+s2}{\PYZdq{}}\PY{p}{,}
                 \PY{n}{xaxis\PYZus{}title} \PY{o}{=} \PY{l+s+s2}{\PYZdq{}}\PY{l+s+s2}{Number of Clusters:}\PY{l+s+s2}{\PYZdq{}}\PY{p}{,}
                 \PY{n}{yaxis\PYZus{}title} \PY{o}{=} \PY{l+s+s2}{\PYZdq{}}\PY{l+s+s2}{Sum of Squared Distance}\PY{l+s+s2}{\PYZdq{}}\PY{p}{,}
                 \PY{n}{showlegend}\PY{o}{=}\PY{k+kc}{False}\PY{p}{)}
\PY{n}{fig}\PY{o}{.}\PY{n}{show}\PY{p}{(}\PY{p}{)}
\end{Verbatim}
\end{tcolorbox}

    
    
    Best Choice : Cluster 3

    \begin{tcolorbox}[breakable, size=fbox, boxrule=1pt, pad at break*=1mm,colback=cellbackground, colframe=cellborder]
\prompt{In}{incolor}{11}{\boxspacing}
\begin{Verbatim}[commandchars=\\\{\}]
\PY{n}{km} \PY{o}{=} \PY{n}{KMeans}\PY{p}{(}\PY{l+m+mi}{3}\PY{p}{)}
\PY{n}{km}\PY{o}{.}\PY{n}{fit}\PY{p}{(}\PY{n}{df1}\PY{p}{[}\PY{l+s+s1}{\PYZsq{}}\PY{l+s+s1}{Temprature}\PY{l+s+s1}{\PYZsq{}}\PY{p}{]}\PY{o}{.}\PY{n}{to\PYZus{}numpy}\PY{p}{(}\PY{p}{)}\PY{o}{.}\PY{n}{reshape}\PY{p}{(}\PY{o}{\PYZhy{}}\PY{l+m+mi}{1}\PY{p}{,}\PY{l+m+mi}{1}\PY{p}{)}\PY{p}{)}
\PY{n}{df1}\PY{o}{.}\PY{n}{loc}\PY{p}{[}\PY{p}{:}\PY{p}{,}\PY{l+s+s1}{\PYZsq{}}\PY{l+s+s1}{Temp Labels}\PY{l+s+s1}{\PYZsq{}}\PY{p}{]} \PY{o}{=} \PY{n}{km}\PY{o}{.}\PY{n}{labels\PYZus{}}
\PY{n}{fig} \PY{o}{=} \PY{n}{px}\PY{o}{.}\PY{n}{scatter}\PY{p}{(}\PY{n}{df1}\PY{p}{,} \PY{l+s+s1}{\PYZsq{}}\PY{l+s+s1}{Date}\PY{l+s+s1}{\PYZsq{}}\PY{p}{,} \PY{l+s+s1}{\PYZsq{}}\PY{l+s+s1}{Temprature}\PY{l+s+s1}{\PYZsq{}}\PY{p}{,} \PY{n}{color}\PY{o}{=}\PY{l+s+s1}{\PYZsq{}}\PY{l+s+s1}{Temp Labels}\PY{l+s+s1}{\PYZsq{}}\PY{p}{)}
\PY{n}{fig}\PY{o}{.}\PY{n}{update\PYZus{}layout}\PY{p}{(}\PY{n}{title} \PY{o}{=} \PY{l+s+s2}{\PYZdq{}}\PY{l+s+s2}{Temprature clusters.}\PY{l+s+s2}{\PYZdq{}}\PY{p}{,}
                 \PY{n}{xaxis\PYZus{}title}\PY{o}{=}\PY{l+s+s2}{\PYZdq{}}\PY{l+s+s2}{Date}\PY{l+s+s2}{\PYZdq{}}\PY{p}{,} \PY{n}{yaxis\PYZus{}title}\PY{o}{=}\PY{l+s+s2}{\PYZdq{}}\PY{l+s+s2}{Temprature}\PY{l+s+s2}{\PYZdq{}}\PY{p}{)}
\PY{n}{fig}\PY{o}{.}\PY{n}{show}\PY{p}{(}\PY{p}{)}
\end{Verbatim}
\end{tcolorbox}

    
    
    Insights: Despite having 4 seasons we can see 3 main clusturs based on
tempratures. Jan, Feb and Dec are the coldest months. Apr, May, Jun,
Jul, Aug and Sep; all have hotter tempratures. Mar, Oct and Nov are the
months that have tempratures neither too hot nor too cold.

    \begin{tcolorbox}[breakable, size=fbox, boxrule=1pt, pad at break*=1mm,colback=cellbackground, colframe=cellborder]
\prompt{In}{incolor}{12}{\boxspacing}
\begin{Verbatim}[commandchars=\\\{\}]
\PY{n}{fig} \PY{o}{=} \PY{n}{px}\PY{o}{.}\PY{n}{histogram}\PY{p}{(}\PY{n}{x}\PY{o}{=}\PY{n}{df1}\PY{p}{[}\PY{l+s+s1}{\PYZsq{}}\PY{l+s+s1}{Temprature}\PY{l+s+s1}{\PYZsq{}}\PY{p}{]}\PY{p}{,} \PY{n}{nbins}\PY{o}{=}\PY{l+m+mi}{200}\PY{p}{,} \PY{n}{histnorm}\PY{o}{=}\PY{l+s+s1}{\PYZsq{}}\PY{l+s+s1}{density}\PY{l+s+s1}{\PYZsq{}}\PY{p}{)}
\PY{n}{fig}\PY{o}{.}\PY{n}{update\PYZus{}layout}\PY{p}{(}\PY{n}{title}\PY{o}{=}\PY{l+s+s1}{\PYZsq{}}\PY{l+s+s1}{Frequency chart of temprature readings:}\PY{l+s+s1}{\PYZsq{}}\PY{p}{,}
                 \PY{n}{xaxis\PYZus{}title}\PY{o}{=}\PY{l+s+s1}{\PYZsq{}}\PY{l+s+s1}{Temprature}\PY{l+s+s1}{\PYZsq{}}\PY{p}{,} \PY{n}{yaxis\PYZus{}title}\PY{o}{=}\PY{l+s+s1}{\PYZsq{}}\PY{l+s+s1}{Count}\PY{l+s+s1}{\PYZsq{}}\PY{p}{)}
\end{Verbatim}
\end{tcolorbox}

    
    
    There is a cluster from 26.2-27.5 and mean temprature for most months
during history has been between 26.8-26.9 \#\#\# Let's see if we can get
some insights from yearly mean temprature data. I am going to treat this
as a time series as well.

    \hypertarget{yearly-average-temperature}{%
\section{Yearly Average Temperature}\label{yearly-average-temperature}}

    \begin{tcolorbox}[breakable, size=fbox, boxrule=1pt, pad at break*=1mm,colback=cellbackground, colframe=cellborder]
\prompt{In}{incolor}{13}{\boxspacing}
\begin{Verbatim}[commandchars=\\\{\}]
\PY{n}{df}\PY{p}{[}\PY{l+s+s1}{\PYZsq{}}\PY{l+s+s1}{Yearly Mean}\PY{l+s+s1}{\PYZsq{}}\PY{p}{]} \PY{o}{=} \PY{n}{df}\PY{o}{.}\PY{n}{iloc}\PY{p}{[}\PY{p}{:}\PY{p}{,}\PY{l+m+mi}{1}\PY{p}{:}\PY{p}{]}\PY{o}{.}\PY{n}{mean}\PY{p}{(}\PY{n}{axis}\PY{o}{=}\PY{l+m+mi}{1}\PY{p}{)} \PY{c+c1}{\PYZsh{}\PYZsh{} Axis 1 for row wise and axis 0 for columns.}
\PY{n}{fig} \PY{o}{=} \PY{n}{go}\PY{o}{.}\PY{n}{Figure}\PY{p}{(}\PY{n}{data}\PY{o}{=}\PY{p}{[}
    \PY{n}{go}\PY{o}{.}\PY{n}{Scatter}\PY{p}{(}\PY{n}{name}\PY{o}{=}\PY{l+s+s1}{\PYZsq{}}\PY{l+s+s1}{Yearly Tempratures}\PY{l+s+s1}{\PYZsq{}} \PY{p}{,} \PY{n}{x}\PY{o}{=}\PY{n}{df}\PY{p}{[}\PY{l+s+s1}{\PYZsq{}}\PY{l+s+s1}{YEAR}\PY{l+s+s1}{\PYZsq{}}\PY{p}{]}\PY{p}{,} \PY{n}{y}\PY{o}{=}\PY{n}{df}\PY{p}{[}\PY{l+s+s1}{\PYZsq{}}\PY{l+s+s1}{Yearly Mean}\PY{l+s+s1}{\PYZsq{}}\PY{p}{]}\PY{p}{,} \PY{n}{mode}\PY{o}{=}\PY{l+s+s1}{\PYZsq{}}\PY{l+s+s1}{lines}\PY{l+s+s1}{\PYZsq{}}\PY{p}{)}\PY{p}{,}
    \PY{n}{go}\PY{o}{.}\PY{n}{Scatter}\PY{p}{(}\PY{n}{name}\PY{o}{=}\PY{l+s+s1}{\PYZsq{}}\PY{l+s+s1}{Yearly Tempratures}\PY{l+s+s1}{\PYZsq{}} \PY{p}{,} \PY{n}{x}\PY{o}{=}\PY{n}{df}\PY{p}{[}\PY{l+s+s1}{\PYZsq{}}\PY{l+s+s1}{YEAR}\PY{l+s+s1}{\PYZsq{}}\PY{p}{]}\PY{p}{,} \PY{n}{y}\PY{o}{=}\PY{n}{df}\PY{p}{[}\PY{l+s+s1}{\PYZsq{}}\PY{l+s+s1}{Yearly Mean}\PY{l+s+s1}{\PYZsq{}}\PY{p}{]}\PY{p}{,} \PY{n}{mode}\PY{o}{=}\PY{l+s+s1}{\PYZsq{}}\PY{l+s+s1}{markers}\PY{l+s+s1}{\PYZsq{}}\PY{p}{)}
\PY{p}{]}\PY{p}{)}
\PY{n}{fig}\PY{o}{.}\PY{n}{update\PYZus{}layout}\PY{p}{(}\PY{n}{title}\PY{o}{=}\PY{l+s+s1}{\PYZsq{}}\PY{l+s+s1}{Yearly Mean Temprature :}\PY{l+s+s1}{\PYZsq{}}\PY{p}{,}
                 \PY{n}{xaxis\PYZus{}title}\PY{o}{=}\PY{l+s+s1}{\PYZsq{}}\PY{l+s+s1}{Time}\PY{l+s+s1}{\PYZsq{}}\PY{p}{,} \PY{n}{yaxis\PYZus{}title}\PY{o}{=}\PY{l+s+s1}{\PYZsq{}}\PY{l+s+s1}{Temprature in Degrees}\PY{l+s+s1}{\PYZsq{}}\PY{p}{)}
\PY{n}{fig}\PY{o}{.}\PY{n}{show}\PY{p}{(}\PY{p}{)}

\PY{n}{fig} \PY{o}{=} \PY{n}{px}\PY{o}{.}\PY{n}{scatter}\PY{p}{(}\PY{n}{df}\PY{p}{,}\PY{n}{x} \PY{o}{=} \PY{l+s+s1}{\PYZsq{}}\PY{l+s+s1}{YEAR}\PY{l+s+s1}{\PYZsq{}}\PY{p}{,} \PY{n}{y} \PY{o}{=} \PY{l+s+s1}{\PYZsq{}}\PY{l+s+s1}{Yearly Mean}\PY{l+s+s1}{\PYZsq{}}\PY{p}{,} \PY{n}{trendline} \PY{o}{=} \PY{l+s+s1}{\PYZsq{}}\PY{l+s+s1}{lowess}\PY{l+s+s1}{\PYZsq{}}\PY{p}{)}
\PY{n}{fig}\PY{o}{.}\PY{n}{update\PYZus{}layout}\PY{p}{(}\PY{n}{title}\PY{o}{=}\PY{l+s+s1}{\PYZsq{}}\PY{l+s+s1}{Trendline Over The Years :}\PY{l+s+s1}{\PYZsq{}}\PY{p}{,}
                 \PY{n}{xaxis\PYZus{}title}\PY{o}{=}\PY{l+s+s1}{\PYZsq{}}\PY{l+s+s1}{Time}\PY{l+s+s1}{\PYZsq{}}\PY{p}{,} \PY{n}{yaxis\PYZus{}title}\PY{o}{=}\PY{l+s+s1}{\PYZsq{}}\PY{l+s+s1}{Temprature in Degrees}\PY{l+s+s1}{\PYZsq{}}\PY{p}{)}
\PY{n}{fig}\PY{o}{.}\PY{n}{show}\PY{p}{(}\PY{p}{)}
\end{Verbatim}
\end{tcolorbox}

    
    
    \begin{Verbatim}[commandchars=\\\{\}]
D:\textbackslash{}anaconda\textbackslash{}lib\textbackslash{}site-packages\textbackslash{}statsmodels\textbackslash{}tools\textbackslash{}\_testing.py:19: FutureWarning:

pandas.util.testing is deprecated. Use the functions in the public API at
pandas.testing instead.

    \end{Verbatim}

    
    
    We can see that the issue of global warning is true.¶ The yearly mean
temprature was not incresing till 1980. It was only after 1979 that we
can see the gradual increse in yearly mean temprature. After 2015,
yearly temprature has incresed drastically. But, There are some problems
in this figure. We are seeing a monthly like up-down pattern in yearly
tempratures as well. This is not understandable. Because with months, we
have a phenominan of seasons and the earth the revolving around sun in a
eliptic path. But this pattern is not expected in yearly temprature.

    \hypertarget{month-wise-temperature-analysis}{%
\section{Month Wise Temperature
Analysis}\label{month-wise-temperature-analysis}}

    \begin{tcolorbox}[breakable, size=fbox, boxrule=1pt, pad at break*=1mm,colback=cellbackground, colframe=cellborder]
\prompt{In}{incolor}{14}{\boxspacing}
\begin{Verbatim}[commandchars=\\\{\}]
\PY{n}{fig} \PY{o}{=} \PY{n}{px}\PY{o}{.}\PY{n}{line}\PY{p}{(}\PY{n}{df1}\PY{p}{,} \PY{l+s+s1}{\PYZsq{}}\PY{l+s+s1}{Year}\PY{l+s+s1}{\PYZsq{}}\PY{p}{,} \PY{l+s+s1}{\PYZsq{}}\PY{l+s+s1}{Temprature}\PY{l+s+s1}{\PYZsq{}}\PY{p}{,} \PY{n}{facet\PYZus{}col}\PY{o}{=}\PY{l+s+s1}{\PYZsq{}}\PY{l+s+s1}{Month}\PY{l+s+s1}{\PYZsq{}}\PY{p}{,} \PY{n}{facet\PYZus{}col\PYZus{}wrap}\PY{o}{=}\PY{l+m+mi}{4}\PY{p}{)}
\PY{n}{fig}\PY{o}{.}\PY{n}{update\PYZus{}layout}\PY{p}{(}\PY{n}{title}\PY{o}{=}\PY{l+s+s1}{\PYZsq{}}\PY{l+s+s1}{Monthly temprature throught history:}\PY{l+s+s1}{\PYZsq{}}\PY{p}{)}
\PY{n}{fig}\PY{o}{.}\PY{n}{show}\PY{p}{(}\PY{p}{)}
\end{Verbatim}
\end{tcolorbox}

    
    
    \hypertarget{seasonwise-analysis}{%
\section{Seasonwise Analysis}\label{seasonwise-analysis}}

    \begin{tcolorbox}[breakable, size=fbox, boxrule=1pt, pad at break*=1mm,colback=cellbackground, colframe=cellborder]
\prompt{In}{incolor}{15}{\boxspacing}
\begin{Verbatim}[commandchars=\\\{\}]
\PY{n}{df}\PY{p}{[}\PY{l+s+s1}{\PYZsq{}}\PY{l+s+s1}{Winter}\PY{l+s+s1}{\PYZsq{}}\PY{p}{]} \PY{o}{=} \PY{n}{df}\PY{p}{[}\PY{p}{[}\PY{l+s+s1}{\PYZsq{}}\PY{l+s+s1}{DEC}\PY{l+s+s1}{\PYZsq{}}\PY{p}{,} \PY{l+s+s1}{\PYZsq{}}\PY{l+s+s1}{JAN}\PY{l+s+s1}{\PYZsq{}}\PY{p}{,} \PY{l+s+s1}{\PYZsq{}}\PY{l+s+s1}{FEB}\PY{l+s+s1}{\PYZsq{}}\PY{p}{]}\PY{p}{]}\PY{o}{.}\PY{n}{mean}\PY{p}{(}\PY{n}{axis}\PY{o}{=}\PY{l+m+mi}{1}\PY{p}{)}
\PY{n}{df}\PY{p}{[}\PY{l+s+s1}{\PYZsq{}}\PY{l+s+s1}{Summer}\PY{l+s+s1}{\PYZsq{}}\PY{p}{]} \PY{o}{=} \PY{n}{df}\PY{p}{[}\PY{p}{[}\PY{l+s+s1}{\PYZsq{}}\PY{l+s+s1}{MAR}\PY{l+s+s1}{\PYZsq{}}\PY{p}{,} \PY{l+s+s1}{\PYZsq{}}\PY{l+s+s1}{APR}\PY{l+s+s1}{\PYZsq{}}\PY{p}{,} \PY{l+s+s1}{\PYZsq{}}\PY{l+s+s1}{MAY}\PY{l+s+s1}{\PYZsq{}}\PY{p}{]}\PY{p}{]}\PY{o}{.}\PY{n}{mean}\PY{p}{(}\PY{n}{axis}\PY{o}{=}\PY{l+m+mi}{1}\PY{p}{)}
\PY{n}{df}\PY{p}{[}\PY{l+s+s1}{\PYZsq{}}\PY{l+s+s1}{Monsoon}\PY{l+s+s1}{\PYZsq{}}\PY{p}{]} \PY{o}{=} \PY{n}{df}\PY{p}{[}\PY{p}{[}\PY{l+s+s1}{\PYZsq{}}\PY{l+s+s1}{JUN}\PY{l+s+s1}{\PYZsq{}}\PY{p}{,} \PY{l+s+s1}{\PYZsq{}}\PY{l+s+s1}{JUL}\PY{l+s+s1}{\PYZsq{}}\PY{p}{,} \PY{l+s+s1}{\PYZsq{}}\PY{l+s+s1}{AUG}\PY{l+s+s1}{\PYZsq{}}\PY{p}{,} \PY{l+s+s1}{\PYZsq{}}\PY{l+s+s1}{SEP}\PY{l+s+s1}{\PYZsq{}}\PY{p}{]}\PY{p}{]}\PY{o}{.}\PY{n}{mean}\PY{p}{(}\PY{n}{axis}\PY{o}{=}\PY{l+m+mi}{1}\PY{p}{)}
\PY{n}{df}\PY{p}{[}\PY{l+s+s1}{\PYZsq{}}\PY{l+s+s1}{Autumn}\PY{l+s+s1}{\PYZsq{}}\PY{p}{]} \PY{o}{=} \PY{n}{df}\PY{p}{[}\PY{p}{[}\PY{l+s+s1}{\PYZsq{}}\PY{l+s+s1}{OCT}\PY{l+s+s1}{\PYZsq{}}\PY{p}{,} \PY{l+s+s1}{\PYZsq{}}\PY{l+s+s1}{NOV}\PY{l+s+s1}{\PYZsq{}}\PY{p}{]}\PY{p}{]}\PY{o}{.}\PY{n}{mean}\PY{p}{(}\PY{n}{axis}\PY{o}{=}\PY{l+m+mi}{1}\PY{p}{)}
\PY{n}{seasonal\PYZus{}df} \PY{o}{=} \PY{n}{df}\PY{p}{[}\PY{p}{[}\PY{l+s+s1}{\PYZsq{}}\PY{l+s+s1}{YEAR}\PY{l+s+s1}{\PYZsq{}}\PY{p}{,} \PY{l+s+s1}{\PYZsq{}}\PY{l+s+s1}{Winter}\PY{l+s+s1}{\PYZsq{}}\PY{p}{,} \PY{l+s+s1}{\PYZsq{}}\PY{l+s+s1}{Summer}\PY{l+s+s1}{\PYZsq{}}\PY{p}{,} \PY{l+s+s1}{\PYZsq{}}\PY{l+s+s1}{Monsoon}\PY{l+s+s1}{\PYZsq{}}\PY{p}{,} \PY{l+s+s1}{\PYZsq{}}\PY{l+s+s1}{Autumn}\PY{l+s+s1}{\PYZsq{}}\PY{p}{]}\PY{p}{]}
\PY{n}{seasonal\PYZus{}df} \PY{o}{=} \PY{n}{pd}\PY{o}{.}\PY{n}{melt}\PY{p}{(}\PY{n}{seasonal\PYZus{}df}\PY{p}{,} \PY{n}{id\PYZus{}vars}\PY{o}{=}\PY{l+s+s1}{\PYZsq{}}\PY{l+s+s1}{YEAR}\PY{l+s+s1}{\PYZsq{}}\PY{p}{,} \PY{n}{value\PYZus{}vars}\PY{o}{=}\PY{n}{seasonal\PYZus{}df}\PY{o}{.}\PY{n}{columns}\PY{p}{[}\PY{l+m+mi}{1}\PY{p}{:}\PY{p}{]}\PY{p}{)}
\PY{n}{seasonal\PYZus{}df}\PY{o}{.}\PY{n}{columns}\PY{o}{=}\PY{p}{[}\PY{l+s+s1}{\PYZsq{}}\PY{l+s+s1}{Year}\PY{l+s+s1}{\PYZsq{}}\PY{p}{,} \PY{l+s+s1}{\PYZsq{}}\PY{l+s+s1}{Season}\PY{l+s+s1}{\PYZsq{}}\PY{p}{,} \PY{l+s+s1}{\PYZsq{}}\PY{l+s+s1}{Temprature}\PY{l+s+s1}{\PYZsq{}}\PY{p}{]}
\end{Verbatim}
\end{tcolorbox}

    \begin{tcolorbox}[breakable, size=fbox, boxrule=1pt, pad at break*=1mm,colback=cellbackground, colframe=cellborder]
\prompt{In}{incolor}{16}{\boxspacing}
\begin{Verbatim}[commandchars=\\\{\}]
\PY{n}{fig} \PY{o}{=} \PY{n}{px}\PY{o}{.}\PY{n}{scatter}\PY{p}{(}\PY{n}{seasonal\PYZus{}df}\PY{p}{,} \PY{l+s+s1}{\PYZsq{}}\PY{l+s+s1}{Year}\PY{l+s+s1}{\PYZsq{}}\PY{p}{,} \PY{l+s+s1}{\PYZsq{}}\PY{l+s+s1}{Temprature}\PY{l+s+s1}{\PYZsq{}}\PY{p}{,} \PY{n}{facet\PYZus{}col}\PY{o}{=}\PY{l+s+s1}{\PYZsq{}}\PY{l+s+s1}{Season}\PY{l+s+s1}{\PYZsq{}}\PY{p}{,} \PY{n}{facet\PYZus{}col\PYZus{}wrap}\PY{o}{=}\PY{l+m+mi}{2}\PY{p}{,} \PY{n}{trendline}\PY{o}{=}\PY{l+s+s1}{\PYZsq{}}\PY{l+s+s1}{ols}\PY{l+s+s1}{\PYZsq{}}\PY{p}{)}
\PY{n}{fig}\PY{o}{.}\PY{n}{update\PYZus{}layout}\PY{p}{(}\PY{n}{title}\PY{o}{=}\PY{l+s+s1}{\PYZsq{}}\PY{l+s+s1}{Seasonal mean tempratures throught years:}\PY{l+s+s1}{\PYZsq{}}\PY{p}{)}
\PY{n}{fig}\PY{o}{.}\PY{n}{show}\PY{p}{(}\PY{p}{)}
\end{Verbatim}
\end{tcolorbox}

    
    
    We can again see a positive trendline between temprature and time. The
trendline does not have a very high positive correlation with years but
still it is not negligable

    \begin{tcolorbox}[breakable, size=fbox, boxrule=1pt, pad at break*=1mm,colback=cellbackground, colframe=cellborder]
\prompt{In}{incolor}{17}{\boxspacing}
\begin{Verbatim}[commandchars=\\\{\}]
\PY{n}{px}\PY{o}{.}\PY{n}{scatter}\PY{p}{(}\PY{n}{df1}\PY{p}{,} \PY{l+s+s1}{\PYZsq{}}\PY{l+s+s1}{Month}\PY{l+s+s1}{\PYZsq{}}\PY{p}{,} \PY{l+s+s1}{\PYZsq{}}\PY{l+s+s1}{Temprature}\PY{l+s+s1}{\PYZsq{}}\PY{p}{,} \PY{n}{size}\PY{o}{=}\PY{l+s+s1}{\PYZsq{}}\PY{l+s+s1}{Temprature}\PY{l+s+s1}{\PYZsq{}}\PY{p}{,} \PY{n}{animation\PYZus{}frame}\PY{o}{=}\PY{l+s+s1}{\PYZsq{}}\PY{l+s+s1}{Year}\PY{l+s+s1}{\PYZsq{}}\PY{p}{)}
\end{Verbatim}
\end{tcolorbox}

    
    
    \hypertarget{forecasting}{%
\section{Forecasting}\label{forecasting}}

    \begin{tcolorbox}[breakable, size=fbox, boxrule=1pt, pad at break*=1mm,colback=cellbackground, colframe=cellborder]
\prompt{In}{incolor}{18}{\boxspacing}
\begin{Verbatim}[commandchars=\\\{\}]
\PY{c+c1}{\PYZsh{}\PYZsh{} I am using decision tree regressor for prediction as the data does not actually have a linear trend.}
\PY{k+kn}{from} \PY{n+nn}{sklearn}\PY{n+nn}{.}\PY{n+nn}{tree} \PY{k}{import} \PY{n}{DecisionTreeRegressor}
\PY{k+kn}{from} \PY{n+nn}{sklearn}\PY{n+nn}{.}\PY{n+nn}{model\PYZus{}selection} \PY{k}{import} \PY{n}{train\PYZus{}test\PYZus{}split} 
\PY{k+kn}{from} \PY{n+nn}{sklearn}\PY{n+nn}{.}\PY{n+nn}{metrics} \PY{k}{import} \PY{n}{r2\PYZus{}score} 

\PY{n}{df2} \PY{o}{=} \PY{n}{df1}\PY{p}{[}\PY{p}{[}\PY{l+s+s1}{\PYZsq{}}\PY{l+s+s1}{Year}\PY{l+s+s1}{\PYZsq{}}\PY{p}{,} \PY{l+s+s1}{\PYZsq{}}\PY{l+s+s1}{Month}\PY{l+s+s1}{\PYZsq{}}\PY{p}{,} \PY{l+s+s1}{\PYZsq{}}\PY{l+s+s1}{Temprature}\PY{l+s+s1}{\PYZsq{}}\PY{p}{]}\PY{p}{]}\PY{o}{.}\PY{n}{copy}\PY{p}{(}\PY{p}{)}
\PY{n}{df2} \PY{o}{=} \PY{n}{pd}\PY{o}{.}\PY{n}{get\PYZus{}dummies}\PY{p}{(}\PY{n}{df2}\PY{p}{)}
\PY{n}{y} \PY{o}{=} \PY{n}{df2}\PY{p}{[}\PY{p}{[}\PY{l+s+s1}{\PYZsq{}}\PY{l+s+s1}{Temprature}\PY{l+s+s1}{\PYZsq{}}\PY{p}{]}\PY{p}{]}
\PY{n}{x} \PY{o}{=} \PY{n}{df2}\PY{o}{.}\PY{n}{drop}\PY{p}{(}\PY{n}{columns}\PY{o}{=}\PY{l+s+s1}{\PYZsq{}}\PY{l+s+s1}{Temprature}\PY{l+s+s1}{\PYZsq{}}\PY{p}{)}

\PY{n}{dtr} \PY{o}{=} \PY{n}{DecisionTreeRegressor}\PY{p}{(}\PY{p}{)}
\PY{n}{train\PYZus{}x}\PY{p}{,} \PY{n}{test\PYZus{}x}\PY{p}{,} \PY{n}{train\PYZus{}y}\PY{p}{,} \PY{n}{test\PYZus{}y} \PY{o}{=} \PY{n}{train\PYZus{}test\PYZus{}split}\PY{p}{(}\PY{n}{x}\PY{p}{,}\PY{n}{y}\PY{p}{,}\PY{n}{test\PYZus{}size}\PY{o}{=}\PY{l+m+mf}{0.3}\PY{p}{)}
\PY{n}{dtr}\PY{o}{.}\PY{n}{fit}\PY{p}{(}\PY{n}{train\PYZus{}x}\PY{p}{,} \PY{n}{train\PYZus{}y}\PY{p}{)}
\PY{n}{pred} \PY{o}{=} \PY{n}{dtr}\PY{o}{.}\PY{n}{predict}\PY{p}{(}\PY{n}{test\PYZus{}x}\PY{p}{)}
\PY{n}{r2\PYZus{}score}\PY{p}{(}\PY{n}{test\PYZus{}y}\PY{p}{,} \PY{n}{pred}\PY{p}{)}
\end{Verbatim}
\end{tcolorbox}

            \begin{tcolorbox}[breakable, size=fbox, boxrule=.5pt, pad at break*=1mm, opacityfill=0]
\prompt{Out}{outcolor}{18}{\boxspacing}
\begin{Verbatim}[commandchars=\\\{\}]
0.9602265589249582
\end{Verbatim}
\end{tcolorbox}
        
    A high r2 value means that our predictive model is working good(For now,
because there is a lot more than just the r\_squared statistic, and we
can't determine how good a model is based only on r2 statistic. But,
that we'll discuss later. ). Now, Let's see the forecasted data for 2018

    \begin{tcolorbox}[breakable, size=fbox, boxrule=1pt, pad at break*=1mm,colback=cellbackground, colframe=cellborder]
\prompt{In}{incolor}{19}{\boxspacing}
\begin{Verbatim}[commandchars=\\\{\}]
\PY{n}{next\PYZus{}Year} \PY{o}{=} \PY{n}{df1}\PY{p}{[}\PY{n}{df1}\PY{p}{[}\PY{l+s+s1}{\PYZsq{}}\PY{l+s+s1}{Year}\PY{l+s+s1}{\PYZsq{}}\PY{p}{]}\PY{o}{==}\PY{l+m+mi}{2017}\PY{p}{]}\PY{p}{[}\PY{p}{[}\PY{l+s+s1}{\PYZsq{}}\PY{l+s+s1}{Year}\PY{l+s+s1}{\PYZsq{}}\PY{p}{,} \PY{l+s+s1}{\PYZsq{}}\PY{l+s+s1}{Month}\PY{l+s+s1}{\PYZsq{}}\PY{p}{]}\PY{p}{]}
\PY{n}{next\PYZus{}Year}\PY{o}{.}\PY{n}{Year}\PY{o}{.}\PY{n}{replace}\PY{p}{(}\PY{l+m+mi}{2017}\PY{p}{,}\PY{l+m+mi}{2018}\PY{p}{,} \PY{n}{inplace}\PY{o}{=}\PY{k+kc}{True}\PY{p}{)}
\PY{n}{next\PYZus{}Year}\PY{o}{=} \PY{n}{pd}\PY{o}{.}\PY{n}{get\PYZus{}dummies}\PY{p}{(}\PY{n}{next\PYZus{}Year}\PY{p}{)}
\PY{n}{temp\PYZus{}2018} \PY{o}{=} \PY{n}{dtr}\PY{o}{.}\PY{n}{predict}\PY{p}{(}\PY{n}{next\PYZus{}Year}\PY{p}{)}

\PY{n}{temp\PYZus{}2018} \PY{o}{=} \PY{p}{\PYZob{}}\PY{l+s+s1}{\PYZsq{}}\PY{l+s+s1}{Month}\PY{l+s+s1}{\PYZsq{}}\PY{p}{:}\PY{n}{df1}\PY{p}{[}\PY{l+s+s1}{\PYZsq{}}\PY{l+s+s1}{Month}\PY{l+s+s1}{\PYZsq{}}\PY{p}{]}\PY{o}{.}\PY{n}{unique}\PY{p}{(}\PY{p}{)}\PY{p}{,} \PY{l+s+s1}{\PYZsq{}}\PY{l+s+s1}{Temprature}\PY{l+s+s1}{\PYZsq{}}\PY{p}{:}\PY{n}{temp\PYZus{}2018}\PY{p}{\PYZcb{}}
\PY{n}{temp\PYZus{}2018}\PY{o}{=}\PY{n}{pd}\PY{o}{.}\PY{n}{DataFrame}\PY{p}{(}\PY{n}{temp\PYZus{}2018}\PY{p}{)}
\PY{n}{temp\PYZus{}2018}\PY{p}{[}\PY{l+s+s1}{\PYZsq{}}\PY{l+s+s1}{Year}\PY{l+s+s1}{\PYZsq{}}\PY{p}{]} \PY{o}{=} \PY{l+m+mi}{2018}
\PY{n}{temp\PYZus{}2018}
\end{Verbatim}
\end{tcolorbox}

            \begin{tcolorbox}[breakable, size=fbox, boxrule=.5pt, pad at break*=1mm, opacityfill=0]
\prompt{Out}{outcolor}{19}{\boxspacing}
\begin{Verbatim}[commandchars=\\\{\}]
   Month  Temprature  Year
0    JAN       20.59  2018
1    FEB       23.08  2018
2    MAR       25.58  2018
3    APR       29.17  2018
4    MAY       30.47  2018
5    JUN       29.44  2018
6    JUL       28.31  2018
7    AUG       28.12  2018
8    SEP       28.11  2018
9    OCT       26.81  2018
10   NOV       23.92  2018
11   DEC       21.47  2018
\end{Verbatim}
\end{tcolorbox}
        
    \begin{tcolorbox}[breakable, size=fbox, boxrule=1pt, pad at break*=1mm,colback=cellbackground, colframe=cellborder]
\prompt{In}{incolor}{20}{\boxspacing}
\begin{Verbatim}[commandchars=\\\{\}]
\PY{n}{forecasted\PYZus{}temp} \PY{o}{=} \PY{n}{pd}\PY{o}{.}\PY{n}{concat}\PY{p}{(}\PY{p}{[}\PY{n}{df1}\PY{p}{,}\PY{n}{temp\PYZus{}2018}\PY{p}{]}\PY{p}{,} \PY{n}{sort}\PY{o}{=}\PY{k+kc}{False}\PY{p}{)}\PY{o}{.}\PY{n}{groupby}\PY{p}{(}\PY{n}{by}\PY{o}{=}\PY{l+s+s1}{\PYZsq{}}\PY{l+s+s1}{Year}\PY{l+s+s1}{\PYZsq{}}\PY{p}{)}\PY{p}{[}\PY{l+s+s1}{\PYZsq{}}\PY{l+s+s1}{Temprature}\PY{l+s+s1}{\PYZsq{}}\PY{p}{]}\PY{o}{.}\PY{n}{mean}\PY{p}{(}\PY{p}{)}\PY{o}{.}\PY{n}{reset\PYZus{}index}\PY{p}{(}\PY{p}{)}
\PY{n}{fig} \PY{o}{=} \PY{n}{go}\PY{o}{.}\PY{n}{Figure}\PY{p}{(}\PY{n}{data}\PY{o}{=}\PY{p}{[}
    \PY{n}{go}\PY{o}{.}\PY{n}{Scatter}\PY{p}{(}\PY{n}{name}\PY{o}{=}\PY{l+s+s1}{\PYZsq{}}\PY{l+s+s1}{Yearly Mean Temprature}\PY{l+s+s1}{\PYZsq{}}\PY{p}{,} \PY{n}{x}\PY{o}{=}\PY{n}{forecasted\PYZus{}temp}\PY{p}{[}\PY{l+s+s1}{\PYZsq{}}\PY{l+s+s1}{Year}\PY{l+s+s1}{\PYZsq{}}\PY{p}{]}\PY{p}{,} \PY{n}{y}\PY{o}{=}\PY{n}{forecasted\PYZus{}temp}\PY{p}{[}\PY{l+s+s1}{\PYZsq{}}\PY{l+s+s1}{Temprature}\PY{l+s+s1}{\PYZsq{}}\PY{p}{]}\PY{p}{,} \PY{n}{mode}\PY{o}{=}\PY{l+s+s1}{\PYZsq{}}\PY{l+s+s1}{lines}\PY{l+s+s1}{\PYZsq{}}\PY{p}{)}\PY{p}{,}
    \PY{n}{go}\PY{o}{.}\PY{n}{Scatter}\PY{p}{(}\PY{n}{name}\PY{o}{=}\PY{l+s+s1}{\PYZsq{}}\PY{l+s+s1}{Yearly Mean Temprature}\PY{l+s+s1}{\PYZsq{}}\PY{p}{,} \PY{n}{x}\PY{o}{=}\PY{n}{forecasted\PYZus{}temp} \PY{p}{[}\PY{l+s+s1}{\PYZsq{}}\PY{l+s+s1}{Year}\PY{l+s+s1}{\PYZsq{}}\PY{p}{]}\PY{p}{,} \PY{n}{y}\PY{o}{=}\PY{n}{forecasted\PYZus{}temp}\PY{p}{[}\PY{l+s+s1}{\PYZsq{}}\PY{l+s+s1}{Temprature}\PY{l+s+s1}{\PYZsq{}}\PY{p}{]}\PY{p}{,} \PY{n}{mode}\PY{o}{=}\PY{l+s+s1}{\PYZsq{}}\PY{l+s+s1}{markers}\PY{l+s+s1}{\PYZsq{}}\PY{p}{)}
\PY{p}{]}\PY{p}{)}
\PY{n}{fig}\PY{o}{.}\PY{n}{update\PYZus{}layout}\PY{p}{(}\PY{n}{title}\PY{o}{=}\PY{l+s+s1}{\PYZsq{}}\PY{l+s+s1}{Forecasted Temprature:}\PY{l+s+s1}{\PYZsq{}}\PY{p}{,}
                 \PY{n}{xaxis\PYZus{}title}\PY{o}{=}\PY{l+s+s1}{\PYZsq{}}\PY{l+s+s1}{Time}\PY{l+s+s1}{\PYZsq{}}\PY{p}{,} \PY{n}{yaxis\PYZus{}title}\PY{o}{=}\PY{l+s+s1}{\PYZsq{}}\PY{l+s+s1}{Temprature in Degrees}\PY{l+s+s1}{\PYZsq{}}\PY{p}{)}
\PY{n}{fig}\PY{o}{.}\PY{n}{show}\PY{p}{(}\PY{p}{)}
\end{Verbatim}
\end{tcolorbox}

    
    
    \begin{tcolorbox}[breakable, size=fbox, boxrule=1pt, pad at break*=1mm,colback=cellbackground, colframe=cellborder]
\prompt{In}{incolor}{ }{\boxspacing}
\begin{Verbatim}[commandchars=\\\{\}]

\end{Verbatim}
\end{tcolorbox}


    % Add a bibliography block to the postdoc
    
    
    
\end{document}
